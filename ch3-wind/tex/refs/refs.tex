\section{References}
%\begin{thebibliography}{1}

%\bibitem{avissar} Avissar, R., and T. Schmidt (1998), An evaluation of the scale at which ground-surface heat flux patchiness affects the convective boundary layer using large-eddy simulations, \textit{Journal of the Atmospheric Sciences}, 55(16), 2666--2689.

%\bibitem{bedard}
%B\'edard, J., W. Yu, Y. Gagnon, and C. Masson (2013), Development of a geophysic model output statistics module for improving short-term numerical wind predictions over complex sites, \textit{Wind Energy}, 16(8), 1131--1147.
%
%\bibitem{carc}
%Carcangiu 2014
%
%\bibitem{carv}
%Carvalho 2012
%
%\bibitem{chen}
%Chen and Avissar 1994
%
%\bibitem{deppe}
%Deppe 2013
%
%\bibitem{draxl}
%Draxl 2014
%
%\bibitem{drusch}
%Drusch 2007
%
%\bibitem{eid}
%Eidenshink, J. C., and J. L. Faundeen (1994). The 1 km AVHRR global land data set: first stages in implementation. \textit{International Journal of Remote Sensing}, 15(17), 3443-3462.
%
%\bibitem{ellis}
%Ellis 2014
%
%\bibitem{fab}
%Fabbri 2005
%
%\bibitem{foley}
%Foley 2012
%
%\bibitem{ge}
%GE Energy, Western wind and solar integration study, May 2010.
%
%\bibitem{ipcc}
%IPCC, 2013: Climate Change 2013: The Physical Science Basis. Contribution of Working Group I to the Fifth Assessment Report of the Intergovernmental Panel on Climate Change [Stocker, T.F., D. Qin, G.-K. Plattner, M. Tignor, S.K. Allen, J. Boschung, A. Nauels, Y. Xia, V. Bex and P.M. Midgley (eds.)]. Cambridge University Press, Cambridge, United Kingdom and New York, NY, USA, 1535 pp.
%
%\bibitem{jac}
%Jacobson, M. Z., and M. A. Delucchi (2011), Providing all global energy with wind, water, and solar power, Part I: Technologies, energy resources, quantities and areas of infrastructure, and materials, \textit{Energy Policy}, 39(3), 1154--1169.
%
%\bibitem{kusiac}
%Kusiac 2009
%
%\bibitem{mans}
%Mansbach 2010
%
%\bibitem{marj}
%Marjanovic 2014
%
%\bibitem{miller}
%Miller 2003
%
%\bibitem{millerwhite}
%Miller and White 1998
%
%\bibitem{mont}
%Monteiro 2009
%
%\bibitem{ncep}
%National Centers for Environmental Prediction/National Weather Service/NOAA/U.S. Department of Commerce (1998), GCIP NCEP Eta model output, http://rda.ucar.edu/datasets/ds609.2/, Research Data Archive at the National Center for Atmospheric Research, Computational and Information Systems Laboratory, Boulder, Colo. (Updated monthly.) Accessed 9 Aug 2014.
%
%\bibitem{ortiz}
%Ortiz-Garcia 2011
%
%\bibitem{phys}
%Physick 1980
%
%\bibitem{pinson}
%Pinson and Madsen 2009
%
%\bibitem{pleim}
%Pleim 2007
%
%\bibitem{porter}
%Porter and Rogers 2010
%
%\bibitem{ran}
%Ranaboldo 2013
%
%\bibitem{rodell}
%Rodell 2004
%
%\bibitem{skam}
%Skamarock 2008
%
%\bibitem{eia}
%U.S. Energy Information Administration (2014), Electric Power Monthly with Data for July 2014, \url{http://www.eia.gov/electricity/monthly/current_year/september2014.pdf}
%
%\bibitem{whart}
%Wharton 2011
%
%\bibitem{zhong}
%Zhong 2014

%\end{thebibliography}
