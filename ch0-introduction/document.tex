\chapter{Introduction}
\label{c.intro}

How does ET (and transpiration especially) matter for the atmosphere?
\begin{itemize}
\item energy and moisture
\item large latent heat of vaporization, ET can consume much of incoming energy - e.g. XX W/m2 or XX\% of incoming solar on a typical day in XX environment.
\item effects on T, q, p, wind
\end{itemize}

How do plants control transpiration?
\begin{itemize}
\item stomatal conductance is actively controlled and responds to XX environmental variables; the Jarvis model and Ball-Berry-Collatz model are widely used, and they approach it from these two different angles...
\item fraction of ET that is transpiration, in different parts of the world and different seasons (Trenberth?  other refs?? that nature paper using isotopes?)
\item limiting factors in different climates, seasonality in different climates
\item species differences (hydraulic strategies)
\end{itemize}

How can ET/transpiration be measured?
\begin{itemize}
\item leaf level, whole tree, flux tower, satellite, catchment water balance - cite ET measurement review paper
\item scale issues...
\end{itemize}

How can the effects of ET on the atmosphere be tested?
\begin{itemize}
\item models (range of complexity) - can systematically change vegetation or surface fluxes to test effects (give examples)
\item observations a la Seneviratne (but can be hard to parse species-specific effects)
\end{itemize}

What we do here:
\begin{itemize}
\item use observations at the tree scale to infer species differences in stomatal response to the environment
\item scale observations up to regional level and compare to regional scale satellite estimates in the Northern California Coast Range
\item test the effect of different stomatal responses on the Northern California atmospheric boundary layer (temperature, humidity, and depth) using both a simple and a complex model
\item test the effects of regional perturbations in soil moisture on near-surface winds, with application to wind energy forecasting
\end{itemize}