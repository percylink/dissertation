\chapter{Effect of regional-scale forest species conversion on the atmospheric boundary layer in the Northern California Coast Range}
\label{c.BL}

\textbf{Abstract:}  Common evergreen tree species in Northern California respond to summer drought with different water use strategies.  In this study, the effect of these different water use patterns on the atmospheric boundary layer is estimated, using two atmospheric models, one simple and one comprehensive.  Two tree species with very different water use strategies are tested, in order to quantify the maximum impact of species distribution on the dry season atmosphere, using extreme regional-scale scenarios of complete forest species conversion from 100\% Douglas fir (\textit{Pseudotsuga menziesii}) to 100\% Pacific madrone (\textit{Arbutus menziesii}).  For representative mid-summer periods, atmospheric boundary layer conditions (temperature, humidity, and boundary layer depth) are compared between a model land surface with Douglas fir stomatal response parameters versus one with Pacific madrone stomatal response parameters.  For both species cases, soil moisture is varied from dry to wet and free-tropospheric conditions are varied from cooler and moister to hotter and drier.  The simple model is a one-dimensional (``slab'') atmospheric boundary layer model that simulates the coupled evolution of the daytime surface energy balance and the growth of the daytime boundary layer by convective entrainment of free-tropospheric air.  The comprehensive model is a three-dimensional regional atmospheric model, the Weather Research and Forecasting (WRF) model.  In both models, when soils are dry, the summertime afternoon mixed layer over the Pacific madrone forest is cooler (by ~1-1.5 deg C), moister (by ~1 g/kg), and shallower (by ~200-500 m) than that over the Douglas fir forest.  The near-surface temperature and humidity differences between the species cases, as simulated in WRF, are even larger: over the madrone forest, the air at 2 m above ground is ~1.5-2.5 deg C cooler and ~2-3 g/kg moister than the air at 2 m above ground over the douglas fir forest.  These results suggest that shifts in species composition of Northern California forests could affect the atmospheric boundary layer in the dry season, and these potential effects should be considered in forest management decisions and assessment of regional climate change impacts.


\section{Introduction}

Because Douglas fir and Pacific madrone respond differently to VPD and soil moisture, these two tree species transpire maximally in different seasons in Northern California, with Douglas fir peaking in spring and Pacific madrone peaking in mid-late summer.  These differences in seasonal water flux may cause the summertime energy partitioning at the land surface to differ between a Douglas-fir-dominated landscape and a Pacific-madrone-dominated landscape.  In this chapter, we use two atmospheric models, one simple and one complex, to estimate the effect of latent heat differences on atmospheric boundary layer temperature, depth, and humidity, in the hypothetical cases of a northern California Coast Range completely dominated by either Douglas fir or Pacific madrone.

The land surface influences the temperature and humidity of the atmospheric boundary layer by several mechanisms, including albedo, surface roughness, and stomatal control of evaporative cooling [\textit{Bonan}, 2008].  Net radiation absorbed by the land surface is partitioned into sensible heat and evapotranspiration (``latent heat"); sensible heat directly warms the atmospheric boundary layer, while latent heat moistens the atmospheric boundary layer but does not increase its temperature if no condensation occurs.  Increased evapotranspiration leads to a cooler, moister, shallower boundary layer, while suppressed evapotranspiration leads to a hotter, drier, deeper boundary layer [\textit{Bonan}, 2008; \textit{Seneviratne et al.}, 2010; \textit{de Arellano et al.}, 2012; \textit{Fischer et al.}, 2007; \textit{Lobell and Bonfils}, 2008; \textit{Mueller and Seneviratne}, 2012; \textit{Durre et al.}, 2000; \textit{Hirschi et al.}, 2010; \textit{Lee et al.}, 2005].  

In forested regions during rain-free periods, the evapotranspiration flux is dominated by transpiration [REF] and thus depends strongly on active stomatal control.  Stomata respond to multiple environmental variables, including root-zone water availability, atmospheric evaporative demand (measured by $VPD$), photosynthetically active radiation, CO$_2$ concentration, and temperature [Jarvis, SELLERS OR BALL-BERRY-COLLATZ?].  Seasonal or anomalous drought most strongly affects root-zone water availability and $VPD$.  Root-zone water supply exerts nonlinear control on $g_s$, with $g_s$ insensitive at high water content but declining nearly linearly below a threshold water content until a minimum water content is reached [FEDDES, CHEN, SOME RODRIGUEZ-ITURBE PAPER?]; the threshold and minimum water contents vary among species [REFS].  $g_s$ declines with increasing $VPD$, also nonlinearly, and species with higher $g_s$ at low $VPD$ show more rapid decline of $g_s$ with increasing $VPD$ [\textit{Oren et al.}, 1999].

Vegetation types with low stomatal conductance can create a hotter, deeper, and drier atmospheric boundary layer.  In boreal forests in summer, needleleaf trees have more conservative stomatal behavior than do broadleaf trees, resulting in lower evapotranspiration, increased sensible heat flux, and higher boundary layer temperature and depth and lower humidity [\textit{Baldocchi et al.}, 2000; \textit{Liu et al.}, 2005].  In temperate Europe, as well, forest and grassland transpiration respond differently to $VPD$: early in the heat wave of 2003 (before depletion of soil moisture), forest sites had lower evapotranspiration and greater sensible heat flux than did grassland sites, due at least in part to greater stomatal closure in forests in response to high $VPD$ [\textit{Teuling et al.}, 2010].  The differences between plant types in partitioning between latent and sensible heat are an important source of uncertainty in modeled land-atmosphere interactions [\textit{Bonan}, 2008; de Noblet-Ducoudre 2012].  ALSO CITE SIQUEIRA AND JUANG AND SWANN AND LEE.

In this chapter, we quantify the effect of species differences in stomatal environmental response on near-surface air temperature, humidity, and boundary layer depth.  We estimate changes in the atmospheric boundary layer between two hypothetical northern California Coast Range forests: one composed entirely of Douglas fir, and the other composed entirely of Pacific madrone.  As shown in Chapter XX, Douglas fir $g_s$ declines below a higher threshold soil moisture content than does Pacific madrone $g_s$.  Additionally, Douglas-fir $g_s$ is high when $VPD$ is low but declines rapidly with increasing $VPD$, whereas Pacific madrone $g_s$ is moderate at low $VPD$ but declines less rapidly with increasing $VPD$.  We use both a simple atmospheric boundary layer model and a complex regional climate model to scale up sap-flow-based observations of the two species's stomatal response to $VPD$ and soil moisture.  By testing extreme scenarios of regional conversion of Northern California forests to all-Douglas-fir or all-Pacific madrone, we estimate the potential differences in atmospheric temperature and humidity resulting from their different stomatal dynamics.


\section{Methods}

We use two atmospheric models to estimate the atmospheric changes: the first is a simple one-dimensional model of a convective boundary layer, and the second is a complex three-dimensional regional climate model.  Because of their differing levels of complexity, these models have complementary strengths and weaknesses.  The simple model isolates the central physical processes of land surface energy partitioning and entrainment of free tropospheric air; however, the simple model neglects secondary but important processes such as lateral advection, topographic effects on flow, and radiation change.  The complex model, on the other hand, includes these and many other processes and can thus represent spatial heterogeneity and unanticipated feedbacks; however, the inclusion of so many processes can obscure the connection between stomatal dynamics and temperature and humidity changes.  By using both models, we test the robustness of the stomatal effects and explore both the central processes and the complex implications.

In both models, we use the stomatal response parameters for soil moisture and $VPD$ derived in the previous chapter to calculate stomatal conductance and thus latent heat flux.  Tests with each model are conducted over a range of soil moisture values and synoptic conditions typical of August in the northern California Coast Range.  We quantify the differences in surface temperature, near-surface air temperature, boundary layer depth, and near-surface humidity between a hypothetical all-Douglas-fir forest and a hypothetical all-Pacific-madrone forest.

\subsection{1-D model}
The 1-D model [\cite{tennekes1981basic}; \textit{Garratt}, 1992 XXXX; \cite{Siqueira:2009qf}] simulates the evolution of boundary layer height, potential temperature, and humidity, given surface fluxes and free troposphere conditions.  The boundary layer is assumed to be well mixed, with uniform potential temperature ($\Theta$, Kelvin) and specific humidity ($Q$, g/kg), and to be capped by a temperature inversion represented by a step change, as shown in Figure 2 from \cite{Siqueira:2009qf}.  The height of the boundary layer, $h$ (m), is assumed to grow due to buoyant convection only, in such a way that the entrainment heat flux at the top of the boundary layer is a fixed fraction of the sensible heat flux at the land surface (as in \textit{Garratt} [1992] XXXX, Section 6.1.5.)  Because the model is 1-D, it assumes horizontal homogeneity, meaning no lateral variation in surface fluxes or properties and no net horizontal advection.

The evolution of $h$ is modeled as

% dh/dt equation
\begin{equation}
\frac{dh}{dt} = (1+2\beta)\frac{H/\rho c_p}{\Gamma_\Theta h},
\label{eqn:dhdt}
\end{equation}
where $H$ is the surface sensible heat flux (W/m$^2$), $\rho$ is the density of air (kg/m$^3$), $c_p$ is the heat capacity of air at constant pressure (J/kg/K), $\Gamma_\Theta$ is the lapse rate of potential temperature above the boundary layer (K/m), and $1+2\beta$ is the proportionality relating surface sensible heat flux to entrainment heat flux at the top of the boundary layer.  The time tendency of the boundary layer height, and thus of entrainment at the top of the boundary layer, is used to solve for the evolution of $\Theta$ and $Q$:

% dTheta/dt equation
\begin{equation}
\frac{d\Theta}{dt} = \frac{1}{h}\left(\frac{H}{\rho c_p}+\Delta\Theta\frac{dh}{dt}\right)
\label{eqn:dTdt}
\end{equation}
% dQ/dt equation
\begin{equation}
\frac{dQ}{dt} = \frac{1}{h}\left(\frac{E}{\rho}+\Delta Q \frac{dh}{dt}\right),
\label{eqn:dQdt}
\end{equation}
where $E$ is surface evapotranspiration (g/m$^2$/s), $\Delta\Theta$ (K) is the jump in potential temperature across the inversion at the top of the mixed layer, and $\Delta Q$ (g/kg) is the jump in specific humidity across the inversion.  These jumps are calculated using

% dDTheta/dt equation
\begin{equation}
\frac{d\Delta\Theta}{dt} = \Gamma_\Theta\frac{dh}{dt}-\frac{d\Theta}{dt}
\label{eqn:dDTdt}
\end{equation}
% dDQ/dt equation
\begin{equation}
\frac{d\Delta Q}{dt} = \Gamma_Q\frac{dh}{dt}-\frac{dQ}{dt},
\label{eqn:dDQdt}
\end{equation}
where $\Gamma_Q$ is the lapse rate of water vapor above the mixed layer.

$E$ is the sum of transpiration ($E_t$) and soil evaporation ($E_{soil}$); evaporation of intercepted canopy water is negligible during the dry season days considered here.  $E_t$ is simulated following the procedure in Section \ref{sec:sapflow_regmeth}: normalized sap velocity at the outer edge of the sapwood ($v_n$, ranging from 0 to 1) is predicted with a Jarvis model for stomatal conductance [\cite{jarvis1976interpretation}] with parameters estimated from sap flow measurements (species-averaged parameters in Table \ref{tbl:sapflow_maxvel}), and $v_n$ is scaled up to regional transpiration using the observed Douglas fir tree-diameter--sapwood-depth relationship (Equation \ref{eqn:sapwood}`) and an FIA-derived tree size distribution [\textit{Woudenberg et al.}, 2010 XXXX, all-species distribution, black line in Figure \ref{fig:sapflow_abundances}].  The Douglas fir sapwood depth relation is used for both the Douglas fir and Pacific madrone model runs in order to eliminate variation due to sapwood area and focus on variation due to stomatal response.

Soil evaporation is estimated using a simplified version of the CLM model soil evaporation scheme [\cite{oleson2010technical}]:
\begin{equation}
E_{soil} = \frac{-\beta_{soi}(q_{air}-q_{ground})}{r_{aw}+r_{litter}},
\end{equation}
where $\beta_{soi}$ is a reduction factor based on soil moisture (Equation 5.68 in \cite{oleson2010technical} with $\theta_{fc,1}=0.15$), $q_{air}$ is the specific humidity of the air (g/kg), $q_{ground}$ is the saturation specific humidity at ground temperature (g/kg), $r_{aw}$ is the resistance to water vapor transfer from the ground to the canopy air space (Equation 5.99 in \cite{oleson2010technical} with $C_s=0.004$ and $u_*=0.4$ m/s), and $r_{litter}$ is the resistance to water vapor transfer through the litter layer (Equation 5.106 in \cite{oleson2010technical} with $L^{eff}_{litter}=0.5$ m$^2$/m$^2$).

Incoming radiation is prescribed using typical values for August in this region.  For incoming solar radiation ($S_{down}$), we use the average diurnal course of total solar radiation measured at an open meadow station at the Angelo Coast Range Reserve (ACRR) on August 15 of 2009-2011.  For downward longwave radiation ($L_{down}$), we use the GEWEX Surface Radiation Budget [Stackhouse \textit{et al.}, 2011 XXXX] mean diurnal pattern from the month of August (years 2003-2007) for the grid cell nearest the ACRR field site.  Shortwave albedo is set to 0.1 and longwave emissivity is set to 0.95 for both species in order to eliminate variation due to vegetation radiative properties (0.1 is the albedo and XX is the emissivity for broadleaf evergreen temperate trees in CLM [\cite{oleson2010technical}].)  Ground heat flux is set equal to 5\% of net radiation [\cite{ogee2001long}].  Aerodynamic resistance ($r_a$) is held constant at 10 s/m, which is a representative value for typical winds and near neutral conditions using Equation 14.33 from \cite{bonan}; this particular value was chosen to give surface and air temperatures close to observations.

Given this predicted $E$, along with prescribed incoming radiation and aerodynamic resistance, the surface energy balance is solved for surface temperature ($T_s$, K) using the Newton-Raphson method and a timestep of 1 second, and $T_s$ is then used to calculate outgoing longwave radiation ($L_{up}$, W/m$^2$) and sensible heat flux ($H$, W/m$^2$).  Potential temperature $\Theta$ is adjusted for altitude to air temperature $T_a$ for calculating $H$ and $LE_t$ (VPD), using an altitude of 400 m and an adiabatic lapse rate of 10 K/km.  

Free troposphere conditions (needed for $\Gamma_{\Theta}$ and $\Gamma_Q$ in Equations \ref{eqn:dhdt}, \ref{eqn:dDTdt}, and \ref{eqn:dDQdt}) are derived from atmospheric soundings at Oakland International Airport, 250 km south of the Rivendell field site (downloaded from the archive at the University of Wyoming, \url{http://weather.uwyo.edu/upperair/sounding.html}).  The sounding site and field site are similar distances from the Pacific coast (16 km for the field site and 25 km for Oakland Airport) and both have prevailing wind directions from the west over the ocean.  Oakland is influenced by fog, but it is also at lower altitude (near sea level), whereas much of northern Coast Range forest region has a base elevation of at least 400 m; as such, we neglect sounding measurements from below 400 m, thus excluding much of the fog.  Profiles of $\Theta$ and $Q$ from 4 AM local time are averaged for the months of July and August from 2009 to 2011, binned by daily maximum temperature ($T_{max}$) measured at the ACRR: cool days ($T_{max} < 20^{\circ}$C), intermediate days ($20^{\circ}$C $\le T_{max} < 30^{\circ}$C), and hot days ($T_{max} \ge 30^{\circ}$C).  The average profiles and the piecewise linear approximations used in the model are shown in Figure \ref{fig:BL_LapseRates}.

\begin{figure}[here]
\includegraphics[width=1\textwidth]{ch2-BL/figures/fitted_lapserates_theta_Q_onefig.png}
\caption{Symbols: potential temperature ($\Theta$, left) and specific humidity ($Q$, right) from Oakland Airport soundings at 04:00 local time, averaged for July and August 2009-2011 and binned by height.  Error bars show one standard deviation.  Lines: piecewise linear approximations to the lapse rates for $\Theta$ and $Q$, which are used in the 1-D boundary layer model.}
\label{fig:BL_LapseRates}
\end{figure}

The range of soil moisture, free troposphere, and tree species conditions tested are listed in Table \ref{table:BL_1Druns}.  

\begin{table}
\begin{tabular}{ l c }
\hline
 & Range of values tested \\ \hline
Jarvis $VPD$ and $\theta_{rel}$ parameters & Douglas fir, Pacific madrone (Table XXX)\\
Lapse rates $\Gamma_{\Theta}$ and $\Gamma_Q$ & 1 (blue in Figure \ref{fig:BL_LapseRates}), 2 (yellow), 3 (red)\\
Relative soil moisture $\theta_{rel}$ & 0.15, 0.2, 0.25, 0.3, 0.35, 0.4, 0.45, 0.5\\
\hline
\end{tabular}
\caption{Range of values tested using the one-dimensional boundary layer model.}
\label{table:BL_1Druns}
\end{table}

\subsection{Regional climate model}
\label{sec:BL_WRFdesc}
In order to further test the impact of these two tree species on the atmospheric boundary layer, we use WRF-Noah [Skamarock \textit{et al.}, 2008], a three-dimensional, non-hydrostatic regional climate model (Weather Research and Forecasting, or WRF) with terrain-following vertical coordinates and a coupled land surface model (Noah).  In WRF, the conservation equations for momentum, mass, and energy are solved numerically to calculate the temporal evolution of atmospheric state variables, including air temperature, pressure, humidity, and wind velocity.  WRF has a range of parameterization options for radiation, turbulence treatment via planetary boundary layer (PBL) schemes or large-eddy simulation closures, cloud microphysics, convection, bottom boundary fluxes of water vapor and heat, and lateral boundary forcing.  

\begin{table}
\begin{tabular}{l l}
\hline
Scheme & Setting \\ \hline
WRF version & 3.6 \\
Grid nesting & two-way \\
Lateral boundary conditions & NCEP Eta analysis \\
Soil levels & 4 \\
Land use and soil categories & USGS \\
Land surface model & Noah \\
Surface layer & MM5 Monin-Obukhov \\
Planetary Boundary Layer (PBL) & ACM2 \\
Microphysics & WSM 3-class simple ice \\
Longwave radiation & RRTM \\
Shortwave radiation & Dudhia \\
Cumulus & Kain-Fritsch (new Eta) \\
Turbulence closure & Horizontal Smagorinzky first order \\
Momentum advection & 5th order horizontal, 3rd order vertical \\
Scalar advection & Positive definite \\
Lateral boundary & 5 grid points \\
\hline
\end{tabular}
\caption{WRF parameterization options.  See Skamarock \textit{et al.} [2008] for description of schemes.}
\label{table:BL_paramschemes}
\end{table}

The parametrization schemes used here are listed in Table \ref{table:BL_paramschemes}.  Most schemes chosen are the default settings for realistic (non-idealized) simulations, with the exception of the ACM2 PBL scheme. The ACM2 scheme is used because of its ability to represent both convective regimes (non-local transport) and shear-dominated regimes (local transport) [Pleim, 2007], and because of its good performance in other WRF studies [CITATIONS including Marjanovic].

\begin{table}
\begin{tabular}{ l c c c c c c c }
\hline
Domain & $\Delta x$ (km) & $\Delta y$ (km) & $nx$ & $ny$ & $nz$ & $\Delta t$ (s) & USGS data res \\ \hline
d01 & 8.1 & 8.1 & 96 & 99 & 45 & 45 & 2 min\\
d02 & 2.7 & 2.7 & 175 & 175 & 45 & 15 & 2 min\\
\hline
\end{tabular}
\caption{Model domains. d01 refers to the outer domain, and d02 refers to the inner domain.}
\label{table:BL_domains}
\end{table}

\begin{figure}[here]
\includegraphics[width=1\textwidth]{ch2-BL/figures/domain_map_cropped.png}
\caption{WRF domains over northern California.  The gray outlines show the domain boundaries; d01 is the outer nest, and d02 is the inner nest.  Left: topographic elevation; right: red shows the test region where vegetation parameters are modified.}
\label{fig:BL_domain}
\end{figure}

The tests are run with two nested domains centered on the northern Coast Range (Figure \ref{fig:BL_domain}).  The use of two nests enables gradual down-scaling of the coarse lateral forcing to the high resolution needed to resolve flow over the Solano wind farm.  We adopt a conservative nesting grid ratio of 3:1.  The outer domain (d01) provides the lateral boundary conditions for the inner domain (d02), and the inner domain states are fed back to the outer domain throughout the region coincident with the inner domain.  Two-way nesting increases model accuracy, particularly in regions of complex terrain [CITATIONS].  The domain resolutions and dimensions are listed in Table \ref{table:BL_domains}.  The lateral boundaries of the outer domain are forced with NCEP Eta 212 grid (40 km) operational analysis [NCEP, 1998] for the period of 2009-08-16 00:00 to 2009-08-30 00:00, with the first 32 hours discarded as model spin-up.  This time period is rain-free and sunny at the Angelo Reserve and represents the mid- to late-summer season when soil is very dry and incoming radiation is still strong (cf. Chapter \ref{c.sapflow} Figure XX - rivendell met time series).

The two hypothetical forests (all-Douglas-fir and all-Pacific-madrone) are tested in the northern Coast Range region highlighted in Figure XX.  In this region, dummy land use and soil types are used, and the VPD and soil moisture stomatal response parameters of this dummy type are modified according to the test case, as described below.  Radiative properties, leaf area, and rooting depth of the dummy type are held constant among the test cases, using the ``Evergreen Needleleaf Forest'' values.  Outside of the test region, observed topography and USGS vegetation and soil types are used.

\begin{table}
\begin{tabular}{ l p{3cm} p{3cm} p{2cm} p{3cm} }
\hline
Vegetation type & $\theta_{ref}$ (m$^3$/m$^3$) & $\theta_{wilt}$ (m$^3$/m$^3$) & $RS$ (s/m) & $HS$ (kg/kg)\\ \hline
%ENF & 0.329 (loam) & 0.066 (loam) & 125 & 47.35\\
Douglas fir & 0.156 & 0.075 & 125 (ENF) & 47.35 (ENF)\\
%EBF & 0.329 (loam) & 0.066 (loam) & 150 & 41.69\\
Pacific madrone & 0.105 & 0.047 & 150 (EBF) & 41.69 (EBF)\\
%Pacific madrone 2 & 0.105 & 0.047 & 300 & 20.\\
\hline
\end{tabular}
\caption{Parameters for Noah's Jarvis formulation of stomatal conductance, by vegetation type.  $RS$ is the Noah minimum stomatal resistance parameter in the Jarvis formulation. $HS$ is the Noah scaling factor for the specific humidity deficit in the Jarvis humidity stress function (Equation XX).  ENF is the USGS Evergreen Needleleaf Forest land use type; EBF is the USGS Evergreen Broadleaf Forest land use type.}
\label{table:BL_NoahJarvisparams}
\end{table}

\begin{table}
\begin{tabular}{ l p{6cm} p{7cm} }
\hline
Run ID & VPD parameters ($RS$, $HS$) & Soil moisture parameters ($\theta_{ref}$, $\theta_{wilt}$)\\ \hline
vDF-sDF & Douglas fir (ENF) & Douglas fir\\
%vEBF-sMD & EBF & Pacific madrone\\
vMD-sMD & Pacific madrone (EBF) & Pacific madrone\\
%vMD2-sMD & Pacific madrone 2 & Pacific madrone\\
\hline
\end{tabular}
\caption{Combinations of stomatal conductance Jarvis parameters used in the WRF tests.  Each pair of parameters is tested for a range of volumetric soil moisture values in the northern Coast Range test region: 0.08, 0.1, 0.12, and 0.14 m$^3$/m$^3$.}
\label{table:BL_WRFruns}
\end{table}

The test region stomatal conductance parameters are modified to quantify the differences between the hypothetical all-Douglas-fir and all-Pacific-madrone cases.  The Noah model uses a Jarvis formulation of stomatal conductance similar to that used in Chapter \ref{c.sapflow} (Equation XX).  The $RS$ parameter is the minimum stomatal resistance (equivalent to $1/g_{s,max}$), and $RS$ is divided by empirical functions of environmental variables to give the actual stomatal resistance.  

The soil moisture stress function is the piecewise-linear, threshold Feddes model [\textit{Feddes et al.}, 1978; \textit{Chen et al.}, 2008].  The parameters for the sigmoid model from Chapter \ref{c.sapflow} (Equation XX, Table XX) thus must be translated to the Feddes parameters (reference or stress point, $\theta_{ref}$, and wilting point, $\theta_{wilt}$).  For each species, we fit a line to $f_{\theta}$ between $f_{\theta}=0.05$ and $f_{\theta}=0.95$ and extrapolate the line to 0 to estimate $\theta_{wilt}$ and to 1 to estimate $\theta_{ref}$ (Figure \ref{fig:BL_FeddesParams}, Table \ref{table:BL_NoahJarvisparams} where $\theta_{rel}$ values are converted to $\theta_{volumetric}$ using $\theta_{max} = 0.439$ m$^3$/m$^3$ for the dominant loam soil type in the test region).

%atmospheric effects of the two tree species are tested by modifying the stomatal conductance parameters for the dummy vegetation and soil types in the test region.

The empirical function representing humidity stress is similar to the asymptotic function used in Chapter \ref{c.sapflow} (Equation \ref{eqn:sapflow_fVPD}):

\begin{equation}
f_{\Delta q} = \frac{1}{1+HS \Delta q},
\label{eqn:BL_WRFq}
\end{equation}
where $\Delta q$ is the difference between saturated specific humidity (kg/kg) and actual specific humidity.

The variation of $1/rsmin * f(\Delta q)$ with $\Delta q$, using USGS parameters for Evergreen Needleleaf Forest (ENF, COLOR) and Evergreen Broadleaf Forest (EBF, COLOR), are shown in Figure XX (top panel).  For comparison, the variation of $g_{s, max}/\alpha * f(VPD)$ with VPD, calculated using sap-flow-derived species averaged parameters for Douglas fir and Pacific madrone (Chapter \ref{c.sapflow}, Table XX), is also shown in Figure XX (bottom panel).  In both the WRF and the sap-flow-derived parameters, the Douglas-fir/ENF case has higher stomatal conductance at low VPD, and the stomatal conductance of Douglas fir and Pacific madrone are close at high VPD.  For the tests presented here, we use the ENF and EBF parameters to represent the species difference in humidity response.  We do not use the sap-flow-derived parameters for several reasons: (1) it is not straightforward to translate the sap-flow-based $g_{s,max}/\alpha$ to $RS$, because $g_{s,max}/\alpha$ refers to stomatal conductance normalized by sapwood area, whereas $RS$ represents resistance on a per-unit-leaf-area basis; (2) $D_o$ is in units of $VPD$ (kPa), while $HS$ is in units of inverse specific humidity ($q$, kg/kg), and the relation between $VPD$ and $q$ varies with temperature; and (3) the atmospheric boundary layer effects when soils are dry depend much more on species differences in soil moisture response than on species differences in humidity response (Figure \ref{fig:BL_testVPDtheta}, below); as such, the impact of errors in humidity response parameters on results from hot summer days is expected to be small.

\begin{figure}[here]
\includegraphics[width=0.9\textwidth]{ch2-BL/figures/theta_params.png}
\caption{Solid lines: sigmoid $f_{\theta}$ functions (Equation XX) for Douglas fir (blue) and Pacific madrone (green) using the species-averaged parameters from Table XX.  Dashed lines: linear regression to $f_{\theta}$ between 0.05 and 0.95.  Symbols: extrapolation of linear regression to $f_{\theta}=0$ and $1$.}
\label{fig:BL_FeddesParams}
\end{figure}


WRF-Noah is run for the all-Douglas-fir and all-Pacific-madrone cases (Tables \ref{table:BL_NoahJarvisparams} and \ref{table:BL_WRFruns}), with a range of soil moisture values (0.08, 0.1, 0.12, 0.14).  These values are equivalent to relative soil moistures of XXXX, given that the saturation moisture content of the loam soil type used in the model is 0.439 m$^3$/m$^3$.  These relative soil moisture values span the range of values observed in August at the Angelo Coast Range Reserve (Chapter \ref{c.sapflow}, Figure XX).  Soil moisture in the Coast Range test region is reset each day at midnight local time, so that the soil moisture deviates only minimally from its stated value (e.g. case vDF-sDF-0.14 has a maximum daily decline of XX m$^3$/m$^3$).


\linespread{1.6}\selectfont

\section{Results}

\subsection{1-D model}

\begin{figure}[here]
\includegraphics[width=1\textwidth]{ch2-BL/figures/testall_Aug15_soilm0pt25_ra10_lapseT2.png}
\caption{}
\label{fig:BL_1Ddiurnal}
\end{figure}

The 1-D model simulates a reasonable diurnal cycle, but with temperatures several degrees higher than observations.  Figure \ref{fig:BL_1Ddiurnal} shows a typical diurnal cycle for lapse rate 2 and relative soil moisture ($\theta_{rel}$) $= 0.25$.  The model run with a Pacific madrone forest has higher transpiration than the run with a Douglas fir forest, because Pacific madrone stomatal conductance is higher at this value of $\theta_{rel}$ (c.f. Figure XX); both cases have very little soil evaporation at this value of $\theta_{rel}$.  As a result, the Pacific madrone case has lower $H$, lower $h$, lower $T_s$ and $T_a$, and higher $Q$ than the Douglas fir case.

\begin{figure}[here]
\includegraphics[width=0.5\textwidth]{ch2-BL/figures/testall_compare_sm_lapse_Ta.png}
\caption{}
\label{fig:BL_1DdiurnalTa}
\end{figure}

\begin{figure}[here]
\includegraphics[width=0.5\textwidth]{ch2-BL/figures/testall_compare_sm_lapse_q.png}
\caption{}
\label{fig:BL_1DdiurnalQ}
\end{figure}

%\clearpage

Figures \ref{fig:BL_1DdiurnalTa} and \ref{fig:BL_1DdiurnalQ} show the dependence of diurnal cycles of $T_a$ and $Q$ on $\theta_{rel}$ and free troposphere conditions.  For both the Pacific madrone case and the Douglas fir case, decreasing $\theta_{rel}$ leads to increasing $T_a$ and decreasing $Q$.  However, the increase in $T_a$ and decrease in $Q$ begin at higher (wetter) values of $\theta_{rel}$ in the Douglas fir case than in the Pacific madrone case.  As expected, the hotter free troposphere conditions resulted in higher $T_a$ (increasing $T_a$ from lapse rate 1 to 2 to 3).  Additionally, the shape of the diurnal cycle differed among the free troposphere cases, with $T_a$ rising most rapidly in the morning for the hottest case (lapse rate 3), resulting from entrainment of high-$\Theta$ air in the steep inversion.  The hotter free troposphere conditions also led to lower $Q$, but with a slower morning decline of $Q$ because of relatively slow boundary layer growth through the steep inversion.

\begin{figure}[here]
\begin{subfigure}{0.5\textwidth}
\includegraphics[width=\textwidth]{ch2-BL/figures/all_afternoon_T.png}
\caption{}
\end{subfigure}
\begin{subfigure}{0.5\textwidth}
\includegraphics[width=\textwidth]{ch2-BL/figures/all_afternoon_Q.png}
\caption{}
\end{subfigure}
\caption{}
\label{fig:BL_345changes}
\end{figure}

%\clearpage

The differences between the species cases are largest in the afternoon; Figure \ref{fig:BL_345changes} illustrates the differences at 3:45 pm in $T_a$ and $Q$ between the Douglas fir and Pacific madrone cases, as a function of $\theta_{rel}$ and free troposphere conditions.  The differences between the Douglas fir and Pacific madrone cases for both $T_a$ and $Q$ are largest at $\theta_{rel}$ values around 0.2, with the Douglas fir case hotter by XX $^o$C and drier by XX g/kg.  A $\theta_{rel}$ value of 0.2-0.25 is typical for the mid- to late-dry-season at the ACRR [\textit{Link et al.}, 2013].  The differences in $T_a$ and $Q$ are somewhat larger for the hottest free troposphere conditions (lapse rate 3).  Interestingly, at $\theta_{rel}$ values higher than about 0.35, the Douglas fir case is actually cooler and moister; such $\theta_{rel}$ values are typical for the late spring and early summer at the ACRR [\textit{Link et al.}, 2013].

%- fraction of moisture from land surface vs. from free troposphere, for different soil moisture / lapse rate / species conditions

\subsection{Regional climate model}
The regional climate model simulates similar temperature and humidity differences as the 1-D model.  Mid-afternoon (XX pm) temperature is warmer in the Douglas-fir case than in the Pacific madrone case by RANGE $^o$C, with the highest temperature difference for an assigned soil moisture of XX (Figure XX); the difference in mid-afternoon is largest at the model level nearest the surface (XX $^o$C) but extends through the lower XX m of the atmosphere with smaller magnitude (XX $^o$C).  Coastal areas in the experimental region (within XX km of the ocean) show a smaller difference between the species cases away from the surface (model levels greater than XX).  There is some downwind transport of the warmer air from the experimental region, especially between 4 pm and 10 pm, when the lower atmosphere over the northern Central Valley is up to XX degrees warmer in the Douglas-fir case than in the Pacific madrone case.

The Douglas-fir case also has lower mid-afternoon specific humidity than the Pacific madrone case water vapor by $\sim$ XX g/kg (Figure XX, top row); again, the difference is largest at an imposed soil moisture value of XX and extends throughout the boundary layer (model layers XX).  The difference is largest in the afternoon and evening (XX pm to XX pm, XX g/kg).  There is again some advection of the drier air (XX g/kg drier in the Douglas-fir case) to the northern Central Valley in the afternoon and evening (XX pm to XX pm).

The hotter and drier boundary layer in the Douglas-fir simulations result from suppressed transpiration and increased sensible heat flux in the Douglas fir case (Figure XX), arising from the greater Douglas-fir stomatal sensitivity to dry soils.   Douglas-fir latent heat flux at 4 pm local time was lower than Pacific madrone transpiration by XX to XX W/m$^2$, and sensible heat flux was higher in the Douglas-fir case by XX to XX W/m$^2$.  The higher sensible heat flux drives greater growth of boundary layer (Figure XX, bottom row), leading to more entrainment of hot, dry free troposphere air.




\section{Discussion and Conclusions}

Large-scale conversion of the northern California Coast Range forest from all-Douglas-fir to all-Pacific-madrone cools and moistens the summertime boundary layer when relative soil moisture is less than 0.3, conditions typical of late summer at the ACRR (Figure \ref{fig:sapflow_met}).   Two atmospheric models, one simple and one complex, simulate a cooling of $\sim$1.5$^\circ$C in the mixed layer ($\sim$2.5$^\circ$C near the surface) and a moistening of 1 g/kg in the mixed layer (2-3 g/kg near the surface).  With 100\% Pacific madrone coverage compared with 100\% Douglas fir coverage, Pacific madrone cools and moistens the boundary layer when soils are dry because madrone stomatal conductance and thus transpiration remains higher at low soil moisture.  The greater transpiration consumes a larger fraction of net radiation in latent heat and reduces the sensible heat flux; because sensible heat is reduced, there is less direct heating of the boundary layer from the surface, and there is less entrainment of the hotter, drier free tropospheric air above the boundary layer.

The simple model and the complex model estimate similar magnitudes of temperature and humidity differences between the Douglas fir and Pacific madrone cases; the similarity of the results confirms the role of evapotranspiration in the near-surface atmosphere in the northern California Coast Range, especially in the dry season.  The 1-D boundary layer model does not include the effects of advection or subsidence and thus overestimates boundary layer height and temperature at the ACRR field site; however, the differences in temperature and humidity between the Douglas fir and Pacific madrone cases simulated by the simple model agree with the differences simulated by the complex model.  Even accounting for the effects of advection, complex topography, and subsidence in WRF, the hotter and drier conditions of the Douglas fir case relative to the Pacific madrone case are a robust result.

The differences in late summer temperature and humidity are due largely to the differences in stomatal response to soil water deficit (Figure \ref{fig:BL_testVPDtheta}).  Importantly, WRF does not represent vegetation-type differences in stomatal response to soil moisture; rather, the $\theta_{ref}$ and $\theta_{wilt}$ parameters depend only on soil type in WRF.  This inability to represent vegetation differences in water stress points prevents WRF from accurately representing the variation in land surface response to drought.  Adding vegetation-type-specific $\theta_{ref}$ and $\theta_{wilt}$ parameters to the WRF Noah land surface model would improve WRF's ability to simulate ecosystem-atmosphere interactions.

While we incorporate sap-flow-derived parameters for stomatal response to soil moisture in both models, in the WRF tests we do not use sap-flow-derived parameters for the $VPD$ response.  In order to incorporate sap-flow-derived $VPD$ parameters into WRF in future work, measurements of the sapwood area to leaf area ratio are necessary (for converting $g_{s,max}$, representing conductance on a per-sapwood-area basis, to $1/RS$, representing stomatal conductance on a per-leaf-area basis).  The sap-flow-based $VPD$ parameters should also be re-estimated using $\Delta q$ instead of $VPD$, in accord with the WRF humidity stress function (Equation \ref{eqn:BL_WRFq}).  Nevertheless, the effect of differences in $VPD$ parameters is small when $VPD$ is high and soil is dry, as is the case in mid- to late-summer (Figure \ref{fig:BL_testVPDtheta}).  The effect of differences in $VPD$ parameters may be larger when soil is wet and $VPD$ is low to moderate, as in winter and spring; simulations of those conditions would require more accurate estimation of species-specific $VPD$ parameters.

Using sap flow measurements to parameterize a regional climate model requires a significant scale jump, from the scale of whole trees and a single hillslope, to the scale of 2.7 km grid boxes and a 500 km x 500 km domain.  Such scale jumps are inherent in the measurement of evapotranspiration; however, further sap flow measurements of these species on slopes with different aspects, elevations, and species mixes are needed in order better to quantify the variability in stomatal response parameters and covariation with other environmental conditions.

The regional cases tested here are extreme scenarios involving the total conversion of the forest from one species (Douglas fir) to the other (Pacific madrone).  However, this may not be wholly unrealistic: Pacific madrones were likely more abundant in the past, due to regular controlled burning by indigenous people [\cite{johnsonACRR}].  It is possible that longer and more severe droughts in a warmer future climate could cause a shift from Douglas fir to Pacific madrone, if fires become more frequent and if Douglas firs are less tolerant of drought.  Moreover, forest management stakeholders are actively discussing controlling the encroachment of Douglas fir in this region [William Dietrich, personal communication]. This study demonstrates that such regional-scale species shifts could have regional-scale impacts on air temperature and humidity in the dry season.

In this study, we integrate tree-scale field observations with physical atmospheric models to test regional atmospheric feedbacks of species-specific stomatal behavior.  We demonstrate the sensitivity of the boundary layer to stomatal dynamics and show that a regional-scale change in dominant evergreen tree species can change summertime afternoon near-surface temperatures by $\sim$2$^\circ$C.  This result underscores the importance of understanding species- and vegetation-type-differences in stomatal response to soil moisture and $VPD$.
