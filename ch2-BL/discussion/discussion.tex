\section{Discussion and Conclusions}

- in WRF, soil water stress does not vary by vegetation type, but instead is a property of the soil type.  However, plant water stress points ($\theta_{ref}$ and $\theta_{wilt}$ in the Feddes terminology) are known to differ among species; models should incorporate vegetation-type-specific $\theta_{ref}$ and $\theta_{wilt}$  to represent more accurately the variation in plant water stress and response to drought.

- state the magnitude of potential effect of species difference on T, Ts, humidity, PBL depth (for each model)

- reiterate that these soil moisture values are typical for the dry season

- magnitude of difference in temperature is large enough to matter for ..., but small compared to synoptic variability

- uncertainty in VPD WRF parameters

- compare the two models: slab ABL does not include effects of advection or subsidence, thus overestimates boundary layer height; even accounting for the effects of advection, complex topography, and subsidence in WRF, the hotter drier conditions of the Douglas-fir case are a robust result.

- acknowledge the scale jumps in using sap flow measurements to parameterize a regional climate model; also emphasize that this is an interesting and novel contribution of this work, to integrate field observations to test atmospheric feedbacks of species-specific stomatal behavior

- acknowledge that these are extreme scenarios - total conversion of forest to one species or the other; however, there is some indication that species composition favored madrones more heavily in the past (fire regime) [REF], and possible that longer and/or more severe droughts in a warmer future climate could cause future shift of species composition.  this study demonstrates that regional-scale species shifts could have regional-scale impacts on near-surface temperature in the dry season.

- also, demonstrates the sensitivity of the boundary layer to stomatal dynamics, and underscores the importance of understanding species- and vegetation-type-differences in stomatal response to VPD and theta

- mention that we are not satisfied with this representation of the subsurface for this geological unit/region

- how Douglas fir suppressed transp might amplify hotter heat waves in future climate (cite Hansen paper on shift of temperature pdf?); transpiration will not increase with increasing VPD, and soils may dry below DF threshold earlier in the summer
