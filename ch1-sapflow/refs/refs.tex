\section*{References}

\bibitem[{\textit{Ackerly}(2004)}]{Ackerly}
Ackerly, D. (2004), Functional strategies of chaparral shrubs in relation to seasonal water deficit and disturbance, 
\textit{Ecol. Monogr.}, \textit{74}(1), 25--44.

\bibitem[{\textit{Aranda}(2000)}]{Aranda}
Aranda, I., L. Gil, and J.~A. Pardos (2000), Water relations and gas exchange in \textit{Fagus sylvatica} L. and \textit{Quercus petraea} (Mattuschka) Liebl. in a mixed stand at their southern limit of distribution in Europe,
\textit{Trees}, \textit{14}, 344--352.

\textit{BALDOCCHI AND XU}

\bibitem[{\textit{Baquedano}(2004)}]{Baquedano}
Baquedano, F.~J., and F.~J. Castillo (2006), Comparative ecophysiological effects of drought on seedlings of the Mediterranean water-saver \textit{Pinus halepensis} and water-spenders \textit{Quercus coccifera} and \textit{Quercus ilex}, 
\textit{Trees}, \textit{20}, 689--700.

\bibitem[{\textit{Black}(1979)}]{Black}
Black, T.~A. (1979), Evapotranspiration from Douglas fir stands exposed to soil water deficits, 
\textit{Water Resour. Res.}, \textit{15}(1), 164--170.

\textbf{BONAN}

\bibitem[{\textit{Bond and Kavanagh}(1999)}]{Bond}
Bond, B.~J., and K.~L. Kavanagh (1999), Stomatal behavior of four woody species in relation to leaf-specific hydraulic conductance and threshold water potential,
\textit{Tree Physiol.}, \textit{19}, 503--510.

\bibitem[{\textit{Brooks}(2006)}]{Brooks}
Brooks, J.~R., F.~C. Meinzer, J.~M. Warren, J.-C. Domec, and R. Coulombe (2006), Hydraulic redistribution in a Douglas-fir forest: lessons from
system manipulations, \textit{Plant, Cell and Environment}, \textit{29}, 138--150.

\bibitem[{\textit{Buckley et al.}(2011)}]{Buckley}
Buckley, T.~N., T.~L. Turnbull, S. Pfautsch, and M.~A. Adams (2011), Nocturnal water loss in mature subalpine \textit{Eucalyptus
delegatensis} tall open forests and adjacent \textit{E. pauciflora}
woodlands, \textit{Ecology and Evolution}, 435--450.

\bibitem[{\textit{Burgess et~al.}(2001)}]{Burgess}
Burgess, S.~S.~O., M.~A. Adams, N.~C. Turner, C.~R. Beverly, C.~K. Ong, A.~A.~H. Khan, and T.~M. Bleby (2001), An improved heat pulse method to measure low and reverse rates of sap flow in woody plants, 
\textit{Tree Physiol.}, \textit{21}, 589--598.

\bibitem[{\textit{\v{C}erm\'{a}k et~al.}(1992)}]{Cermak}
\v{C}erm\'{a}k, J., E. Cienciala, J. Ku\v{c}era, and J.-E. H\"{a}llgren (1992), Radial velocity profiles of water flow in trunks of Norway spruce and oak and the response of spruce to severing, 
\textit{Tree Physiol.}, \textit{10}, 367--380.

\bibitem[{\textit{Chen}(2008)}]{Chen}
Chen, X., Y. Rubin, S. Ma, and D. Baldocchi (2008), Observations and stochastic modeling of soil moisture control on evapotranspiration in a Californian oak savanna,
\textit{Water Resour. Res.}, \textit{44}.

\bibitem[{\textit{Chirino}(2011)}]{Chirino}
Chirino, E., J. Bellot, and J.~R. S\'anchez (2011), Daily sap flow rate as an indicator of drought avoidance mechanisms in five Mediterranean perennial species in semi-arid southeastern Spain,
\textit{Trees}, \textit{25}, 593--606.

\bibitem[{\textit{Choat et~al.}(2012)}]{Choat}
Choat, B., S. Jansen, T.~J. Bodribb, H. Cochard, S. Delzon, R. Bhaskar, S. Bucci, T.~S. Feild, S.~M. Gleason, U.~G. Hacke, A.~L. Jacobsen, F. Lens, H. Maherali, J. Mart\'{i}nez-Vilalta, S. Mayr, M. Mencuccini, P.~J. Mitchell, A. Nardini, J. Pittermann, R.~B. Pratt, J.~S. Sperry, M. Westoby, I.~J. Wright, and A.~E. Zanne (2012), Global convergence in the vulnerability of forests to drought, 
\textit{Nature}, \textit{491}, 752--756.

\bibitem[{\textit{Cohen et~al.}(1985)}]{Cohen}
Cohen, Y., F.~M. Kelliher, and T.~A. Black (1985), Determination of sap flow in Douglas-fir trees using the heat pulse technique, 
\textit{Can. J. For. Res.}, \textit{15}, 422--428.

\textbf{CONSTANTZ AND MURPHY}

\bibitem[{\textit{Dang et~al.}(1997)}]{Dang}
Dang, Q.-L., H.~A. Margolis, M.~R. Coyea, M. Sy, and G.~J. Collatz (1997), Regulation of branch-level gas exchange of boreal trees: roles of shoot water potential and vapor pressure difference, 
\textit{Tree Physiol.}, \textit{17}, 521--535.

\bibitem[{\textit{David}}]{David}
David, T.~S., M.~O. Henriques, C. Kurz-Besson, J. Nunes, F. Valente, M. Vaz, J.~S. Pereira, R. Siegwolf, M.~M. Chaves, L.~C. Gazarini, and J.~S. David (2007), Water-use strategies in two co-occurring Mediterranean evergreen oaks: surviving the summer drought,
\textit{Tree Physiol.}, \textit{27}, 793--803.

\bibitem[{\textit{Dragoni}}]{Dragoni}
Dragoni, D., K.~K. Caylor, H.~P. Schmid (2009), Decoupling structural and environmental determinants of sap velocity, Part II. Observational application,
\textit{Agricultural and Forest Meteorology}, \textit{149}, 570--581.

\bibitem[{\textit{Feddes}}]{Feddes}
Feddes, R., P. Kowalik, and H. Zaradny (1978), \textit{Simulation of Field Water Use and Crop Yield}, John Wiley, New York.

\bibitem[{\textit{Ford et~al.}(2004)}]{Ford}
Ford, C.~R., M.~A. McGuire, R.~J. Mitchell, and R.~O. Teskey (2004), Assessing variation in the radial profile of sap flux density in \textit{Pinus} species and its effect on daily water use,
\textit{Tree Physiol.}, \textit{24}, 241--249.

\bibitem[{\textit{Franks}(2007)}]{Franks}
Franks, P.~J., P.~L. Drake, and R.~H. Froend (2007), Anisohydric but isohydrodynamic: seasonally constant plant water potential gradient explained by a stomatal control mechanism incorporating variable plant hydraulic conductance,
\textit{Plant, Cell and Environment}, \textit{30}, 19--30.

\bibitem[{GLOBE}]{GLOBE}
GLOBE Task Team and others (Hastings, D.~A., P.~K. Dunbar, G.~M. Elphingstone, M. Bootz, H. Murakami, H. Maruyama, H. Masaharu, P. Holland, J. Payne, N.~A. Bryant, T.~L. Logan, J.-P. Muller, G. Schreier, and J.~S. MacDonald), eds. (1999), The Global Land One-kilometer Base Elevation (GLOBE) Digital Elevation Model, Version 1.0. National Oceanic and Atmospheric Administration, National Geophysical Data Center, 325 Broadway, Boulder, Colorado 80303, U.S.A. Digital data base on the World Wide Web (URL: http://www.ngdc.noaa.gov/mgg/topo/globe.html) and CD-ROMs.

\bibitem[{\textit{Granier}(1987)}]{Granier}
Granier, A. (1987), Evaluation of transpiration in a Douglas-fir stand by means of sap flow measurements, 
\textit{Tree Physiol.}, \textit{3}, 309--320.

\bibitem[{\textit{Humphreys et~al.}(2003)}]{Humphreys}
Humphreys, E.~R., T.~A. Black, G.~J. Ethier, G.~B. Drewitt, D.~L. Spittlehouse, E.-M. Jork, Z. Nesic, and N.~J. Livingston (2003), Annual and seasonal variability of sensible and latent heat fluxes above a coastal Douglas-fir forest, British Columbia, Canada, 
\textit{Agricultural and Forest Meteorology}, \textit{115}, 109--125.

\bibitem[{\textit{Jarvis}(1976)}]{Jarvis}
Jarvis, P.~G. (1976), The interpretation of the variations in leaf water potential and stomatal conductance found in canopies in the field, 
\textit{Philos. Trans. R. Soc., B}, \textit{273}, 593--610.

\bibitem[{\textit{Jassal et~al.}(2009)}]{Jassal}
Jassal, R.~S., T.~A. Black, D.~L. Spittlehouse, C. Br\"{u}mmer, and Z. Nesic (2009), Evapotranspiration and water use efficiency in different-aged Pacific Northwest Douglas-fir stands, 
\textit{Agricultural and Forest Meteorology}, \textit{149}, 1168--1178.

\bibitem[{\textit{Johnson}(1979)}]{Johnson}
Johnson, S.~G. (1979), The land-use history of the Coast Range Preserve, Mendocino County, California, 
M.A. thesis, 258 pp., San Francisco State University, San Francisco, May.

\bibitem[{\textit{Jung et~al.}(2010)}]{Jung}
Jung, M., M. Reichstein, P. Ciais, S.~I. Seneviratne, J. Sheffield, M.~L. Goulden, G. Bonan, A. Cescatti, J. Chen, R. de Jeu, A.~J. Dolman, W. Eugster, D. Gerten, D. Gianelle, N. Gobron, J. Heinke, J. Kimball, B.~E. Law, L. Montagnani, Q. Mu, B. Mueller, K. Oleson, D. Papale, A.~D. Richardson, O. Roupsard, S. Running, E. Tomelleri, N. Viovy, U. Weber, C. Williams, E. Wood, S. Zaehle, and K. Zhang (2010), Recent decline in the global land evapotranspiration trend due to limited moisture supply, 
\textit{Nature}, \textit{467}, 951--954.

\bibitem[{\textit{Kumagai}(2012)}]{Kumagai}
Kumagai, T., and A. Porporato (2012), Strategies of a Bornean tropical rainforest water use as a function of rainfall regime: isohydric or anisohydric? 
\textit{Plant, Cell and Environment}, \textit{35}, 61--71.

\bibitem[{\textit{Lindroth and Halldin}(1986)}]{Lindroth}
Lindroth, A., and S. Halldin (1986), Numerical analysis of pine forest evaporation and surface resistance, 
\textit{Agricultural and Forest Meteorology}, \textit{38}, 59--79.

\bibitem[{\textit{Lohammar et~al.}(1980)}]{Lohammar}
Lohammar, T., S. Larsson, S. Linder, and S.~O. Falk (1980), FAST: Simulation models of gaseous exchange in Scots Pine, 
in Structure and Function of Northern Coniferous Forests: An Ecosystem Study, Ecological Bulletins, no. 32, edited by T. Persson,
pp. 505-523, Stockholm.

\bibitem[{\textit{Lorenz}(1956)}]{Lorenz}
Lorenz, E.~N. (1956), Empirical orthogonal functions and statistical weather prediction, 
Scientific Report No. 1, Statistical Forecasting Project, 49 pp., Massachusetts Institute of Technology, Department of Meteorology, Cambridge, MA.

\bibitem[{\textit{Marshall and Waring}(1984)}]{Marshall}
Marshall, J.~D., and R.~H. Waring (1984), Conifers and broadleaf species: stomatal sensitivity differs in western Oregon,
\textit{Can. J. For. Res.}, \textit{14}(6), 905--908.

\bibitem[{\textit{Martinez-Vilalta2}(2003)}]{Martinez-Vilalta2}
Mart\'inez-Vilalta, J., M. Mangir\'on, R. Ogaya, M. Sauret, L. Serrano, J. Pe\~nuelas, and J. Pi\~nol (2003), Sap flow of three co-occurring Mediterranean woody species under varying atmospheric and soil water conditions,
\textit{Tree Physiol.}, \textit{23}, 747--758.

\bibitem[{\textit{Martinez-Vilalta}(2004)}]{Martinez-Vilalta}
Mart\'inez-Vilalta, J., A. Sala, and J. Pi\~nol (2004), The hydraulic architecture of Pinaceae -- a review,
\textit{Plant Ecology}, \textit{171}, 3--13.

\bibitem[{\textit{McDowell et~al.}(2008)}]{McDowell}
McDowell, N., W.~T. Pockman, C.~D. Allen, D.~D. Breshears, N. Cobb, T. Kolb, J. Plaut, J. Sperry, A. West, D.~G. Williams, and E.~A. Yepez (2008), Mechanisms of plant survival and mortality during drought: why do some plants survive while others succumb to drought?, 
\textit{New Phytologist}, \textit{178}, 719--739.

\bibitem[{\textit{Morrow and Mooney}(1974)}]{Morrow}
Morrow, P.~A., and H.~A. Mooney (1974), Drought adaptations in two Californian evergreen sclerophylls, 
\textit{Oecologia}, \textit{15}, 205--222.

\bibitem[{\textit{Mu}(2007)}]{Mu}
Mu, Q., F.~A. Heinsch, M. Zhao, and S.~W. Running (2007), Development of a global evapotranspiration algorithm based on MODIS and global meteorology data, 
\textit{Remote Sensing of Environment}, \textit{111}(4), 519--536.

\bibitem[{\textit{Nadezhdina et~al.}(2002)}]{Nadezhdina}
Nadezhdina, N., J. \v{C}erm\'{a}k, and R. Ceulemans (2002), Radial patterns of sap flow in woody stems of dominant and understory species: scaling errors associated with positioning of sensors, 
\textit{Tree Physiol.}, \textit{22}, 907--918.

\bibitem[{Oleson}]{Oleson}
Oleson, K.~W., D.~M. Lawrence, G.~B. Bonan, M.~G. Flanner, E. Kluzek, P.~J. Lawrence, S. Levis, S.~C. Swenson, P.~E. Thornton, A. Dai, M. Decker, R. Dickinson, J. Feddema, C.~L. Heald, F. Hoffman, J.-F. Lamarque, N. Mahowald, G.-Y. Niu, T. Qian, J. Randerson, S. Running, K. Sakaguchi, A. Slater, R. St\"{o}ckli, A. Wang, Z.-L. Yang, X. Zeng, and X. Zeng (2010), Technical Description of version 4.0 of the Community Land Model (CLM), NCAR Technical Note NCAR/TN-478+ STR, National Center for Atmospheric Research, 257 pp.

\bibitem[{\textit{Oren et~al.}(1999)}]{Oren}
Oren, R., J.~S. Sperry, G.~G. Katul, D.~E. Pataki, B.~E. Ewers, N. Phillips, and K.~V.~R. Sch\"{a}fer (1999), Survey and synthesis of intra- and interspecific variation in stomatal sensitivity to vapour pressure deficit, 
\textit{Plant, Cell and Environment}, \textit{22}, 1515--1526.

\bibitem[{\textit{Oshun}(2012)}]{Oshun}
Oshun, J., D. Rempe, P. Link, K.~A. Simonin, W.~E. Dietrich, T.~E. Dawson, and I. Fung (2012), A look deep inside a hillslope reveals a structured heterogeneity of isotopic reservoirs and distinct water use strategies for adjacent trees, \textit{AGU Fall Meeting Abstracts}, San Francisco, CA.

\bibitem[{\textit{Pataki et~al.}(2000)}]{Pataki}
Pataki, D.~E., R. Oren, and W.~K. Smith (2000), Sap flux of co-occurring species in a western subalpine forest during seasonal soil drought, 
\textit{Ecology}, \textit{81}(9), 2557--2566.

\bibitem[{\textit{Patil et~al.}(2010)}]{Patil}
Patil, A., D. Huard, and C.~J. Fonnesbeck (2010), PyMC: Bayesian stochastic modelling in python,
\textit{J. Stat. Software}, \textit{35}(4), 1--81.

\bibitem[{Raftery}]{Raftery}
Raftery, A.~E., and S.~M. Lewis (1995), The number of iterations, convergence diagnostics, and generic metropolis algorithms, in \textit{Practical Markov Chain Monte Carlo}, W.~R. Gilks, D.~J. Spiegelhalter, and S. Richardson, eds.

\bibitem[{\textit{Rempe et~al.}(2010)}]{Rempe}
Rempe, D., J. Oshun, W. Dietrich, R. Salve, and I. Fung (2010), Controls on the weathering front
depth on hillslopes underlain by mudstones and sandstones, \textit{AGU Fall Meeting Abstracts}, \textit{1},
05.

\bibitem[{\textit{Rodriguez-Iturbe et~al.}(2001)}]{Rodriguez}
Rodriguez-Iturbe, I., A. Porporato, F. Laio, and L. Ridolfi (2001), Intensive or extensive use of soil moisture: plant strategies to cope with stochastic water availability, 
\textit{Geophysical Research Letters}, \textit{28}(23), 4495--4497.

\bibitem[{\textit{Running}(1976)}]{Running}
Running, S.~W. (1976), Environmental control of leaf water conductance in conifers, 
\textit{Can. J. For. Res.}, \textit{6}, 104--112.

\bibitem[{\textit{Salve}(2012)}]{Salve}
Salve, R., D.~M. Rempe, and W.~E. Dietrich (2012), Rain, rock moisture dynamics, and the rapid response of perched groundwater in weathered, fractured argillite underlying a steep hillslope, 
\textit{Water Resour. Res.}, \textit{48}, W11528, doi:10.1029/2012WR012583.

\textbf{SCHULZE 1985}

\bibitem[{\textit{Schwinning}(2010)}]{Schwinning}
Schwinning, S. (2010), The ecohydrology of roots in rocks,
\textit{Ecohydrology}, \textit{3}, 238--245.

\bibitem[{\textit{Schwinning2}(2013)}]{Schwinning2}
Schwinning, S. (2013), Do we need new rhizosphere models for rock-dominated landscapes?,
\textit{Plant Soil}, \textit{362}, 25--31.

\bibitem[{\textit{Sivia}(2006)}]{Sivia}
Sivia, D.~S., and J. Skilling (2006), \textit{Data Analysis: A Bayesian Tutorial}, Oxford University Press, Oxford, UK.

\bibitem[{\textit{Smith}(1966)}]{Smith}
Smith, J.~H.~G., J. Walters, and R.~W. Wellwood (1966), Variation in sapwood thickness of Douglas-fir in relation to tree and section characteristics, \textit{Forest Science}, \textit{1}(12), 97--103.

\bibitem[{\textit{Tan}(1976)}]{Tan}
Tan, C.~S., and T.~A. Black (1976), Factors affecting the canopy resistance of a Douglas-fir forest, 
\textit{Boundary-Layer Meteorology}, \textit{10}, 475--488.

\bibitem[{\textit{Teuling}(2010)}]{Teuling}
Teuling, A.~J., S.~I. Seneviratne, R. St\"{o}ckli, M. Reichstein, E. Moors, P. Ciais, S. Luyssaert, B. van den Hurk, C. Ammann, C. Bernhofer, E. Dellwik, D. Gianelle, B. Gielen, T. Gr\"{u}nwald, K. Klumpp, L. Montagnani, C. Moureaux, M. Sottocornola, and G. Wohlfahrt (2010), Contrasting response of European forest and grassland energy exchange to heatwaves, 
\textit{Nature Geoscience}, \textit{3}, 722--727.

\bibitem[{\textit{Topp}(1980)}]{Topp}
Topp, G.~C., J.~L. Davis, and A.~P. Annan (1980), Electromagnetic determination of soil water content: Measurements in coaxial transmission lines, 
\textit{Water Resour. Res.}, \textit{16}(3), 574--582.

\bibitem[{\textit{USDA}(2005)}]{USDA}
USDA (2005), CALVEG zones and alliances: vegetation descriptions.  \url{http://www.fs.usda.gov/Internet/FSE_DOCUMENTS/fsbdev3_046448.pdf}

\bibitem[{\textit{Vinukollu}}]{Vinukollu}
Vinukollu, R.~K., E.~F. Wood, C.~R. Ferguson, and J.~B. Fisher (2011), Global estimates of evapotranspiration for climate studies using multi-sensor remote sensing data: Evaluation of three process-based approaches,
\textit{Remote Sensing of Environment}, \textit{115}, 801--823.

\bibitem[{\textit{Wang}(1995)}]{Wang}
Wang, Z.~Q., M. Newton, and J.~C. Tappeiner II (1995), Competitive relations between Douglas-fir and Pacific madrone on shallow soils in a Mediterranean climate, 
\textit{Forest Science}, \textit{41}(4), 744--757.

\bibitem[{\textit{Waring}(2011)}]{Waring1}
Waring, R.~H., and J.~J. Landsberg (2011), Generalizing plant--water relations to landscapes, 
\textit{Journal of Plant Ecology}, \textit{4}(1-2), 101--113.

\bibitem[{\textit{Waring}(1978)}]{Waring2}
Waring, R.~H., and S.~W. Running (1978), Sapwood water storage: its contribution to transpiration and effect upon water conductance through the stems of old-growth Douglas-fir, 
\textit{Plant, Cell and Environment}, \textit{1}, 131--140.

\bibitem[{\textit{Warren}(2007)}]{Warren}
Warren, J.~M., F.~C. Meinzer, J.~R. Brooks, J.-C. Domec, and R. Coulombe (2007), Hydraulic redistribution of soil water in two old-growth coniferous forests: quantifying patterns and controls,
\textit{New Phytologist}, \textit{173}, 753--765.

\bibitem[{Williams}]{Williams}
Williams, C., M. Menne, and J. Lawrimore (2012), Modifications to Pairwise Homogeneity Adjustment software to address coding errors and improve run-time efficiency, NCDC Technical Report NCDC No. GHCNM-12-02, National Climatic Data Center, 28 pp.

\bibitem[{\textit{Woudenberg}(2010)}]{Woudenberg}
Woudenberg, S.~W., B.~L. Conkling, B.~M. O'Connell, E.~B. LaPoint, J.~A. Turner, and K.~L. Waddell (2010), The Forest Inventory and Analysis Database: Database description and users manual version 4.0 for Phase 2, Gen. Tech. Rep. RMRS-GTR- 245. Fort Collins, CO: U.S. Department of Agriculture, Forest Service, Rocky Mountain Research Station. 336 p.

\bibitem[{\textit{Zwieniecki}(1995)}]{Zwieniecki}
Zwieniecki, M.~A., and M. Newton (1995), Roots growing in rock fissures: Their morphological adaptation, 
\textit{Plant and Soil}, \textit{172}, 181--187.

\bibitem[{\textit{Zwieniecki}(1996)}]{Zwieniecki2}
Zwieniecki, M.~A., and M. Newton (1996), Seasonal pattern of water depletion from soil--rock profiles in a Mediterranean climate in southwestern Oregon, 
\textit{Can. J. For. Res.}, \textit{26}, 1346--1352.
