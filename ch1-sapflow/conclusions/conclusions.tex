\section{Conclusions}
Two evergreen tree species common to forests of the northern Pacific US coast have different seasons of peak transpiration, due to their different responses to atmospheric evaporative demand and soil water limitation.  Douglas-fir transpiration is phase-shifted from the annual cycle of solar radiation toward an earlier season of transpiration, with higher transpiration in the wet spring; this is because Douglas-firs' stomatal conductance is sensitive to water availability and \textit{VPD} and their transpiration thus declines through the dry season.  Pacific madrone transpiration, in contrast, is phase-shifted toward a later season of transpiration, with higher transpiration in the dry summer; this is because broadleaf tree species at this site, especially Pacific madrones, are less sensitive to water stress and maintain greater stomatal conductance at high \textit{VPD}.

The observations of sap flow were combined with a regional forest inventory to construct a bottom-up estimate of regional transpiration.  This estimate highlights the regional-scale impact of needleleaf evergreen stomatal sensitivity to water stress.  The resulting suppression of dry season transpiration could create feedbacks from the forest to atmospheric temperature and humidity, and the nature of these feedbacks would depend on species distribution [Chapter \ref{c.BL}].  Better constraints on historical and future changes in Pacific coast forest species composition are needed in order to understand the resulting impacts on the land-atmosphere exchange of water and energy.
